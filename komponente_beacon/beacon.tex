\section{Funktionsumfang}
Die Beacon Komponente besteht aus zwei Teilen. Der erste Teil ist der Beacon selbst, auf den sich Android Geräte via Wifi Direct verbinden sollen und der zweite Teil ist eine Android Softwarekomponente, die es ermöglicht auf den Beacon zuzugreifen und diesen für Apps nutzbar zu machen, ohne dass diese sich näher mit dem Verbindungsaufbau und Datenaustausch beschäftigen müssen. Geräte sollen sich automatisch mit den Beacons verbinden und alle auf den Beacons platzierten Dateien, welche sie anfragen, erhalten können. Es soll auch die Möglichkeit geben Dateien auf Beacons hochzuladen, dies soll aber nur durch Administratoren erfolgen können. Diese Komponente soll später in die Ortungs/Schnitzeljagd-App integriert werden. Dort können die bestehenden Import und Export Funktionen genutzt werden um Schnitzeljagden und Ortungsdaten von einem Beacon zu erhalten bzw. diese über einen Beacon zu verteilen.

\section{WiFi Direct}
Wi-Fi Direct, auch Wi-Fi P2P ist ein Protokoll zur Verbindung von Zwei oder mehr WLAN-Geräten ohne einen dedizierten Wireless Access Piont(WAP).
Statt dessen wird ein Software Access Point verwendet, welcher auf einem der teilnehmenden WLAN-Geräte läuft. Dieser Teilnehmer wird auch als Group Owner (GO) bezeichnet und übernimmt sämtliche Koordination der Wi-Fi Direct Gruppe. In einer WiFi-Direct Gruppe muss nur der Group Owner den WiFi Direct Standard unterstützen. Im Kontext des Projektes bietet WiFi-Direct den Vorteil, dass sich das Gerät des Nutzers zum Verbindungsaufbau nicht zu einem HotSpot verbinden muss, sondern im normalen WLAN bleiben kann. So kann ein störungsfreier Datenaustausch stattfinden, ohne dass der Nutzer eingreifen muss. Auch muss das Smartphone des Nutzers keine WiFi Direct Unterstützung haben, wenn der Beacon in der Rolle des Group Owners eingesetzt wird.

\section{Umsetzung}
Für den Beacon wird ein Raspberry Pi Zero W verwendet, auf dem Raspbian installiert ist.
Zunächst ist sicherzustellen, dass der Raspberry Pi Zero W WiFi Direct unterstützt. Dies kann man durch den \enquote{iw list} Befehl in Erkenntnis bringen. Wifi Direct wird dann unterstützt, wenn unter den \enquote{Supported interface modes}, \enquote{P2P-client} und \enquote{P2P-GO} gelistet werden. Da es im Wifi Direct Standard möglich ist Geräte ohne direkte Unterstützung als P2P-client zu nutzen und der Raspberry Pi Zero W P2P-client und P2P-GO unterstützt, sollten auch Wifi Geräte kompatibel sein, welche WiFi Direct nicht unterstützen.
Wifi Direct definiert zwei mögliche Rollen: P2P-client ist ein Wifi client der sich auf einen sogenannten software access point verbindet. Dieser wird von der zweiten möglichen Rolle zur verfügung gestellt. P2p-go(group owner) stellt einen Software-AP zur Verfügung auf den sich mehrere Clients gleichzeitig verbinden können. Diese Clients formen dann eine sogenannte Group. Es ist also wünschenswert den Rasperry Pi als Group Owner zu Konfigurieren.

Um unter Raspbian Zugriff auf WiFi Direct Funktionalitäten zu erhalten, muss ein neues Netzwerk-interface erstellt werden. Dies ist unter Raspbian durch die Nutzung des WPA-supplicant Services möglich. Dazu wird die Konfigurationsdatei von WPA-supplicant modifiziert und der Service neu gestartet. Nach dem Neustart gibt es zwei wesentliche Möglichkeiten mit dem Service zu kommunizieren. Erstens über eine C/C++ Schnittstelle und zweitens über wpa-cli(command line interface). Zunächst wurde wpa-cli zur Kommunikation verwendet, dies bringt jedoch Nachteile mit sich die später erläutert werden. Im Folgenden wird die Aushandlung einer WiFi P2P Verbindung über das Command line interface näher erläutert. Der Befehl \enquote{p2p-find} startet einen discovery Prozess. Das Gerät sucht nach weiteren Peers und macht sich selbst für andere Peers sichtbar. Hat das Gerät andere Peers gefunden kann es eine Verbindungsanfrage stellen. Wichtige Parameter sind dabei: go-intent: 0-15 Beim normalen Verbindungsaufbau wird der Group Owner ausgehandelt. Je höher der go-intent, desto wahrscheinlicher, dass das Gerät Group Owner wird. AuthMethod: die Art und Weise, wie die Verbindung authentifiziert wird. PIN/PBC sind die wichtigsten Möglichkeiten, wobei PIN sich noch weiter unterteilen lässt und die Eingabe oder Verifikation einer PIN erfordert. PBC(push button connect) öffnet ein Zeitfenster, in dem sich ein benanntes Gerät mit dem Gerät das den PBC ausführt verbinden kann. Da sich alle Geräte auf den Pi verbinden können sollen, wird dort jedes anfragende Gerät mit PBC für die Verbindung authentifiziert. Leider funktioniert der Verbindungsaufbau für den Pi als Group Owner nicht, da es im command line interface einen Fehler im WPS(WiFi Protected Setup) gibt. Dies lässt sich eventuell durch Nutzung einer C/C++ Schnittstelle beheben.

Da mit der Aktuellen Methode der Raspberry Pi als Group Owner nicht funktioniert wird beim Verbindungsaufbau vom Pi ein go-intent von 0 übergeben. Damit wird garantiert, dass der pi ein p2p-client wird. Sobald die Verbindung steht, kann mit einer Address Resolution Protocol (ARP) Anfrage die IP der über P2P verbundenen Geräte ermittelt werden. Ab diesem Punkt kann eine normale IP basierte Verbindung begonnen werden und es müssen keine gesonderten Maßnahmen bezüglich P2P ergriffen werden. Unter Android ist der Verbindungsaufbau wesentlich einfacher, da es dort bereits die Klasse WifiP2pConfig gibt, die entsprechende Methoden zur Verfügung stellt.
Im letzten Schritt baut eine normale Client Server Anwendung auf dieser Verbindung auf. Diese kümmert sich um den Datenaustausch, der gewöhnlich über TCP abläuft. Das laden von Dateien auf den Beacon ist zusätzlich gesichert. Bei der ersten Verbindung generiert der Beacon ein Schlüsselpaar, und schickt dem Client einen privaten Schlüssel, mit dem dieser alle Übertragungen an den Beacon signieren kann. Verwendet wird dazu ein DSA Key Pair.