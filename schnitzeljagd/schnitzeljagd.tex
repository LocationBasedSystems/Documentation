\section{Spielprinzip}
\subsection{Schnitzeljagd spielen}
In der Applikation muss zunächst der Menüpunkt Schnitzeljagd ausgewählt werden. Dort sieht der Benutzer eine Liste von allen Schnitzeljagden, die er auf seinem Gerät gespeichert hat. Falls der Benutzer sich dazu entscheidet, eine Schnitzeljagd spielen zu wollen muss er einmal auf den Listeneintrag tippen. Daraufhin öffnet sich die StartPaperchaseActivity, welche Namen und Beschreibung der Schnitzeljagd und Informationen zum Startpunkt darstellt. Um die Schnitzeljagd zu starten muss sich der Benutzer zunächst zum Startpunkt bewegen, erst dann wird der Knopf mit der Aufschrift \enquote{Beginne Schnitzeljagd} aktiviert. Um zum Startpunkt zu gelangen kann auch die Schaltfläche \enquote{Navigation} genutzt werden, welche auf die angepasste RouteFinderActivity weiterleitet und dem Nutzer, falls möglich, eine Route zum Startpunkt anzeigt. Sobald der Benutzer am Startpunkt angelangt ist und auf den \enquote{Beginne Schnitzeljagd}-Knopf geklickt hat, startet die Schnitzeljagd und die PlayPaperchaseActivity öffnet sich. In dieser Activity ist der Hinweistext zu sehen, wie man zum nächsten Ort gelangt. Falls der Hinweis ein Foto enthält, wird dieses zunächst klein und unscharf dargestellt, bis der Benutzer es angeklickt hat, um es auf das volle Bild zu vergrößern. Neben den Hinweisen befindet sich auch ein Textfeld mit der aktuellen Position und aktuell auch ein Debug-Knopf, falls der Ort nicht erreichbar ist. Hat der Benutzer alle Orte und am Ende das Ziel erreicht, bekommt er die an den Zielpunkt hinzugefügte Nachricht und seine benötigte Zeit zu sehen und kann anschließend wieder zur Schnitzeljagdauswahl zurückkehren.

\subsection{Schnitzeljagd erstellen}
Neben der Liste an verfügbaren Schnitzeljagden hat der Benutzer auch die Möglichkeit eine eigene Schnitzeljagd zu erstellen. Dazu muss der FloatingActionButton gedrückt werden, woraufhin sich die AddPaperchaseActivity öffnet, welche die Eingabe des Namens und der Beschreibung der Schnitzeljagd ermöglicht. Zum hinzufügen von Hinweisen muss wieder ein FloatingActionButton gedrückt werden, woraufhin eine Liste an allen gespeicherten Orten (Nodes) angezeigt wird. Der Benutzer kann sich mithilfe der Suchleiste oder durch scrollen einige Orte aussuchen, an denen er seine Hinweispunkte haben möchte. Sobald er bestätigt hat, öffnet sich wieder die AddPaperchaseActivity, in der die Orte in einer DragListView angezeigt werden \cite{woxblom_magnus_draglistview_2014} . Durch langes drücken auf einen Hinweis, kann dieser in der Liste verschoben werden und so kann die Reihenfolge der Schnitzeljagd, sowie Start- und Zielpunkt festgelegt werden. Außerdem können einzelne Hinweise wieder entfernt werden, in dem der Benutzer einen Listeneintrag zur Seite wegwischt. Bevor die Schnitzeljagd erstellt werden kann, muss an jeden Hinweispunkt noch ein Hinweistext, welcher durch antippen hingezugefügt werden kann. Optional besteht die Möglichkeit an seinen Hinweispunkt noch ein Rätsel in Form eines Bildes anzuhängen, da so die Schnitzeljagden um einiges komplexer und interessanter werden können. Zusätzlich zum Erstellen von Schnitzeljagden kann der Benutzer durch langes tippen auf eine Schnitzeljagd diese auch wieder löschen oder bearbeiten.

\section{Umsetzung}
\subsection{Clue}
Ein \enquote{Clue} im Sinne der Anwendung ist das Datenmodel für einen Hinweispunkt im Schnitzeljagdmodell, in der App auch \enquote{Hinweis} genannt. Er repräsentiert einen Ort, an dem für eine Schnitzeljagd ein Hinweis plaziert wurde. Er enthält den Hinweistext und einen Verweis auf den Node, an dem sich der Hinweis befindet. Außerdem ist es möglich ein Foto als Hinweis zu verwenden, dafür kann der optionale Bildpfad verwendet werden.


\subsection{Paperchase}
Ein \enquote{Paperchase} im Sinne der Anwendung ist das Datenmodel für eine gesamte Schnitzeljagd, in der App auch \enquote{Schnitzeljagd} genannt. Ein Objekt vom Typ \enquote{Paperchase} enthält sowohl den Namen und eine kurze Beschreibung der Schnitzeljagd, als auch eine sortierte Liste von mindestens zwei Objekten vom Typ \enquote{Clue}.

\subsection{Persistenz}
Damit neu erstellte oder kürzlich heruntergeladene Schnitzeljagden auch nach dem Beenden der App noch vorhanden sind, wird eine Persistenz-Lösung benötigt. Aufbauend auf die vorhandene Lösung für die Nodes und Edges, wird für die Speicherung der Schnitzeljagd auch eine SQLite-Datenbank verwendet, welche über die PaperchaseDatabaseHandlerFactory erreichbar ist und alle notwendigen Methoden enthält, die zum Management der Datenbank erforderlich sind. Da jeder \enquote{Clue} immer zu einer bestimmten Schnitzeljagd gehört, werden sie nie einzelnd in die Datenbank gespeichert, sondern immer nur sobald die zugehörige Schnitzeljagd gespeichert oder geladen wird.
Da die Datenbank im Applikationsverzeichnis der Anwendung liegt, wird sie im Falle einer Deinstallation gelöscht. Deshalb die Import-/Exportfunktion für die Schnitzeljagd dadurch erweitert worden, dass eine weitere Datei in das Verzeichnis "/IndoorPositioning/Exported" abgespeichert oder daraus geladen werden kann.

Diese Import- und Exportfunktion bietet außerdem eine hervorragende Schnittstelle für die Beaconkommunikation, da so das Speichern und Laden der Daten schon sichergestellt ist.